\documentclass[letterpaper]{article}
\usepackage{graphicx}
\usepackage{amsmath}
\usepackage[utf8]{inputenc}
\usepackage[spanish]{babel}
\usepackage{babelbib}
\usepackage{lmodern}
\usepackage[T1]{fontenc}
\usepackage{color}
\usepackage{framed}
\usepackage{hyperref}
\usepackage{listings}
\usepackage{newtxmath,newtxtext}
\definecolor{red}{RGB}{219,0,0}
\definecolor{pink}{RGB}{255,100,100}
\definecolor{gray}{RGB}{100,100,100}
\lstset{
		basicstyle=\ttfamily,
		frame=single,
		keywordstyle=\color{red},
		commentstyle=\color{gray},
		stringstyle=\color{pink},
		tabsize=3,
		language=verilog,
		backgroundcolor=\color{white}}

\usepackage{fancyhdr}
\pagestyle{fancy}
\usepackage{lastpage}
\lhead{Propuesta Proyecto}
\chead{}
\rhead{Estructuras abstractas de datos y algoritmos}
\lfoot{}
\cfoot{}
\rfoot{\footnotesize Page \thepage\ of \pageref{LastPage}}

\renewcommand{\headrulewidth}{0.4pt}
\renewcommand{\footrulewidth}{0.4pt}
\renewcommand{\footrulewidth}{0.4pt}

\graphicspath{{../media/}}	%%multimedia path
\setlength{\parindent}{0pt}
%%***********************************************************************
\begin{document}

\title{Proyecto final\\Estructuras de Datos abstractos y Algoritmos}
\author{
 Marco Antonio Montero Chavarría Carné: A94000\\
 Juacinho Ze Pelao du Rivaul
 Emilinho du Nazemento
}
\maketitle

\newpage

\section*{Objetivo General}
Implementar una librería en C++ con la capacidad de generar un horario de cursos
de acuerdo a los planes de estudio de Escuela de Ingeniería Eléctrica de la
Universidad de Costa Rica.

\section*{Objetivos específicos}
\begin{itemize}
\item Generar la estructura para una base de datos que contenga la información referente a profesores y cursos.
\item Asignar a partir de la base de datos, profesores a los cursos tomando en cuenta limitaciones de horario de los mismos y cantidad de grupos por curso.
\item Obtener un horario en el cual los cursos de cada bloque tenga al menos una posibilidad de ser matriculados todos sin que allá un traslape de horarios entre ellos.

\end{itemize}


\section*{Justificación}
 En la escuela de ingeniería eléctrica de la universidad de Costa Rica, el proceso de asignación de horarios a los cursos se realiza de una forma semi-automática por medio de una serie de hojas de datos en Excel, esto es útil para el profesor encargado pero tiene ciertas fallas que podrían ser solucionadas, además cabe destacar que el algoritmo puede ser implementado por un lenguaje como C++ y optimizar así el proceso . Un trabajo como este brindará una opción más automatizada que ayudará a agilizar el proceso de creación de horarios, el cual es bastante laborioso en el estado actual. Además ayudará a los integrantes del proyecto a profundizar en conocimientos del lenguaje de programación C++ y un inicio no despreciable en el uso de bases de datos, ya que el gran bagaje del trabajo se basa en trabajar con los datos de una base de datos que puede ser modificada según las necesidades semestrales de la escuela.

\end{document}
